\documentclass[paper=a4, fontsize=11pt]{scrartcl} % A4 paper and 11pt font size

%----------------------------------------------------------------------------------------
%	PACKAGES
%----------------------------------------------------------------------------------------
\usepackage[T1]{fontenc} % Use 8-bit encoding that has 256 glyphs
\usepackage{fourier} % Use the Adobe Utopia font for the document - comment this line to return to the LaTeX default
\usepackage[english]{babel} % English language/hyphenation
\usepackage{amsmath,amsfonts,amsthm} % Math packages
\usepackage{sectsty} % Allows customizing section commands
\usepackage{fancyhdr} % Custom headers and footers
\usepackage{xcolor} % Allows 0-255 RGB values
\usepackage{tabularx, outlines, framed, varwidth, enumitem, graphicx, listings, color, qtree, float, subcaption, newfloat, pdfpages}
\usepackage[left=0.5in, right=0.5in, top=3in, bottom=.25in]{geometry}
\geometry{}

%----------------------------------------------------------------------------------------
%	SET CUSTOMIZATIONS AND FUNCTIONS
%----------------------------------------------------------------------------------------
\sectionfont{\centering \normalfont\scshape} % Make all sections centered, the default font and small caps
\pagestyle{fancyplain} % Makes all pages in the document conform to the custom headers and footers
\fancyhead{} % No page header - if you want one, create it in the same way as the footers below
\fancyfoot[L]{} % Empty left footer
\fancyfoot[C]{} % Empty center footer
\fancyfoot[R]{\thepage} % Page numbering for right footer
\renewcommand{\headrulewidth}{0pt} % Remove header underlines
\renewcommand{\footrulewidth}{0pt} % Remove footer underlines
\setlength{\headheight}{0pt} % Customize the height of the header

\DeclareFloatingEnvironment[fileext=lod]{diagram}

\numberwithin{equation}{section} % Number equations within sections (i.e. 1.1, 1.2, 2.1, 2.2 instead of 1, 2, 3, 4)
\numberwithin{figure}{section} % Number figures within sections (i.e. 1.1, 1.2, 2.1, 2.2 instead of 1, 2, 3, 4)
\numberwithin{table}{section} % Number tables within sections (i.e. 1.1, 1.2, 2.1, 2.2 instead of 1, 2, 3, 4)

\graphicspath{{./figures/}}
%\setlength\parindent{0pt} % Removes all indentation from paragraphs - comment this line for an assignment with lots of text

\makeatletter
	\newcommand*\variableheghtrulefill[1][.4\p@]
	{%
		\leavevmode
		\leaders \hrule \@height #1\relax \hfill
		\null
	}
\makeatother

\definecolor{solBase03}{RGB}{000,043,054}
\definecolor{solBase02}{RGB}{007,054,066}
\definecolor{solBase01}{RGB}{088,110,117}
\definecolor{solBase00}{RGB}{101,123,131}
\definecolor{solBase0}{RGB}{131,148,150}
\definecolor{solBase1}{RGB}{147,161,161}
\definecolor{solBase2}{RGB}{238,232,213}
\definecolor{solBase3}{RGB}{253,246,227}
\definecolor{solYellow}{RGB}{181,137,000}
\definecolor{solOrange}{RGB}{203,075,022}
\definecolor{solRed}{RGB}{220,050,047}
\definecolor{solMagenta}{RGB}{211,054,130}
\definecolor{solViolet}{RGB}{108,113,196}
\definecolor{solBlue}{RGB}{038,139,210}
\definecolor{solCyan}{RGB}{042,161,152}
\definecolor{solGreen}{RGB}{133,153,000}

\lstdefinestyle{mystyle}{
	% To Match
	sensitive=true,	
	%
	% Add border
	frame=lines,
	%
	% Add Margin
	xleftmargin=\parindent,
	%
	% Put extra space under caption
 belowcaptionskip=1\baselineskip,
 %
 % Colors:
	backgroundcolor=\color{solBase3},
 basicstyle=\color{solBase00}\footnotesize,
 keywordstyle=\color{solCyan},
 commentstyle=\color{solBase1},
 stringstyle=\color{solBlue},
 numberstyle=\color{solViolet},
 identifierstyle=\color{solBase00},
 %
 % Formatting Options
 breakatwhitespace=false,
 breaklines=true, % break long lines
 captionpos=b,
 keepspaces=true,
 numbers=left,
 numbersep=5pt,
 showspaces=false,
 showstringspaces=false,
 showtabs=false,
 tabsize=4
}

\lstdefinestyle{smallstyle}{
	% To Match
	sensitive=true,	
	%
	% Add border
	frame=lines,
	%
	% Add Margin
	xleftmargin=\parindent,
	%
	% Put extra space under caption
 belowcaptionskip=1\baselineskip,
 %
 % Colors:
	backgroundcolor=\color{solBase3},
 basicstyle=\color{solBase00}\scriptsize,
 keywordstyle=\color{solCyan},
 commentstyle=\color{solBase1},
 stringstyle=\color{solBlue},
 numberstyle=\color{solViolet},
 identifierstyle=\color{solBase00},
 %
 % Formatting Options
 breakatwhitespace=false,
 breaklines=false, % break long lines
 captionpos=b,
 keepspaces=true,
 numbers=left,
 numbersep=5pt,
 showspaces=false,
 showstringspaces=false,
 showtabs=false,
 tabsize=2
}
 
\lstset{style=mystyle}

%----------------------------------------------------------------------------------------
%	USEFUL COMMANDS
%----------------------------------------------------------------------------------------
%	\makebox[\textwidth][c]{\includegraphics[width=.7\paperwidth]{p2-table}}

%	\newgeometry{top=.75in, bottom=.75in, left=.25in,right=.25in}
%	\newgeometry{top=.75in, bottom=.75in, left=1.25in,right=1.25in}

%	\lstinputlisting[firstline=0, language=C, style=mystyle]{CMPSC360_Homework.cpp}

%\Tree
%	[.<root> [.<left> ][.<middle> ][.<right> ]]

%----------------------------------------------------------------------------------------
%	TITLE SECTION
%----------------------------------------------------------------------------------------

\newcommand{\horrule}[1]{\rule{\linewidth}{#1}} % Create horizontal rule command with 1 argument of height
% \title{Template: Homework 1}
\title{	
\normalfont \normalsize 
%\textsc{Rutgers University, Real Analysis I} \\ [25pt] % Your university, school and/or department name(s)
\horrule{0.5pt} \\[0.4cm] % Thin top horizontal rule
\huge STAT 461: Homework 9 \\ % The assignment title
\horrule{2pt} \\[0.5cm] % Thick bottom horizontal rule
}

\author{\textbf{\underline{Name:}}Kyle Salitrik | \textit{\textbf{\underline{ID\#:}} 997543474} | \textit{\textbf{\underline{PSU ID:}} kps168}} % Your name

\date{\normalsize\today} % Today's date or a custom date

\begin{document}

\maketitle % Print the title

%----------------------------------------------------------------------------------------
%	PROBLEM 1
%----------------------------------------------------------------------------------------
\newgeometry{top=.75in, bottom=.75in, left=1.25in,right=1.25in}
\section*{\variableheghtrulefill[.25ex]\quad Problem 1 \quad\variableheghtrulefill[.25ex]}
\begin{tabularx}{\textwidth}{| X | X | X | X |}
\hline
\textbf{Factor} & \textbf{Fixed/Random} & \textbf{Nested/Crossed} & \textbf{Levels} \\ \hline
State (s) & Random & Crossed w/ type & NY, PA, VT\\ \hline
Type (t) & Fixed & Crossed w/ state & Ag, Forest \\ \hline
Lake (l) & Random & Replicate & 1,2,3,4 \\ \hline
\end{tabularx}

\begin{flalign*}
	Y_{stl} &= \mu + \alpha_{s} + \beta_{t} + \epsilon_{stl} & \\
	\epsilon_{stl} & \sim N(0,\sigma^2) &\\
	\alpha_{s} & \sim N(0,\sigma_{state}^2) &
\end{flalign*}

%----------------------------------------------------------------------------------------
%	PROBLEM 2
%----------------------------------------------------------------------------------------
\section*{\variableheghtrulefill[.25ex]\quad Problem 2 \quad\variableheghtrulefill[.25ex]}
\begin{tabularx}{\textwidth}{| X | X | X | X |}
\hline
\textbf{Factor} & \textbf{Fixed/Random} & \textbf{Nested/Crossed} & \textbf{Levels} \\ \hline
Type (t) & Fixed & Crossed with paint & Mountain, City \\ \hline
Road (r) & Random & Nested in Type & 1, 2, 3, 4, 5, 6, 7, 8, 9, 10 \\ \hline
Paint (p) & Fixed & Crossed with road & Reflective, Non-Reflective \\ \hline
Year (y) & Random & Replicate & 1,2,3 \\ \hline
\end{tabularx}

\begin{flalign*}
	Y_{trpy} &= \mu + \alpha_{t} + \beta_{r(t)} + \gamma_{p} + \left(\alpha\beta\right)_{tp} +   \epsilon_{trpy} & \\
	\epsilon_{ijt} & \sim N(0,\sigma^2) &\\
	\beta_{r(t)} & \sim N(0,\sigma_{road}^2) &
\end{flalign*}

%----------------------------------------------------------------------------------------
%	PROBLEM 3
%----------------------------------------------------------------------------------------
\section*{\variableheghtrulefill[.25ex]\quad Problem 3 \quad\variableheghtrulefill[.25ex]}
\begin{tabularx}{\textwidth}{| X | X | X | X |}
\hline
\textbf{Factor} & \textbf{Fixed/Random} & \textbf{Nested/Crossed} & \textbf{Levels} \\ \hline
Store (s) & Random & Crossed with display & 1,2,3,4 \\ \hline
Display (r) & Fixed & Crossed with store & Yes, No \\ \hline
Week (w) & Random & Replicate & 1,2,3,4 \\ \hline
\end{tabularx}

\begin{flalign*}
	Y_{srw} &= \mu + \alpha_{s} + \beta_{r} + \epsilon_{srw} & \\
	\epsilon_{srw} & \sim N(0,\sigma^2) &\\
	\alpha_{s} & \sim N(0,\sigma_{store}^2) &
\end{flalign*}

\lstinputlisting[language=, caption=Fixed and Random Effect ANOVA]{listings/hw9_p3_anova.txt}
\lstinputlisting[language=, caption=Pairwise Results for Display]{listings/hw9_p3_pairwise.txt}

Looking at the ANOVA tables, we can see that both the fixed and random factors are significant. While performing a pairwise comparison on store is irrelevant, we can look at the difference using a display makes. Because we have a negative value for "No - Yes", we can conclude that the average sales while using a display are higher than average sales without a display.

\newpage
\section*{Code Appendix}
\lstinputlisting[language=R]{461_hw9.R}

\end{document}