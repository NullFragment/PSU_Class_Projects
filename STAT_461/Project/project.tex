\documentclass[paper=a4, fontsize=11pt]{scrartcl} % A4 paper and 11pt font size

%----------------------------------------------------------------------------------------
%	PACKAGES
%----------------------------------------------------------------------------------------
\usepackage[T1]{fontenc} % Use 8-bit encoding that has 256 glyphs
\usepackage{fourier} % Use the Adobe Utopia font for the document - comment this line to return to the LaTeX default
\usepackage[english]{babel} % English language/hyphenation
\usepackage{amsmath,amsfonts,amsthm} % Math packages
\usepackage{sectsty} % Allows customizing section commands
\usepackage{fancyhdr} % Custom headers and footers
\usepackage{xcolor} % Allows 0-255 RGB values
\usepackage{tabularx, outlines, framed, varwidth, enumitem, graphicx, listings, color, qtree, float, subcaption, newfloat, pdfpages, hyperref}
\usepackage[left=0.5in, right=0.5in, top=3in, bottom=.25in]{geometry}
\geometry{}

\hypersetup{
    colorlinks=true,
    linkcolor=blue,
    filecolor=magenta,      
    urlcolor=blue,
}

%----------------------------------------------------------------------------------------
%	SET CUSTOMIZATIONS AND FUNCTIONS
%----------------------------------------------------------------------------------------
\sectionfont{\centering \normalfont\scshape} % Make all sections centered, the default font and small caps
\pagestyle{fancyplain} % Makes all pages in the document conform to the custom headers and footers
\fancyhead{} % No page header - if you want one, create it in the same way as the footers below
\fancyfoot[L]{} % Empty left footer
\fancyfoot[C]{} % Empty center footer
\fancyfoot[R]{\thepage} % Page numbering for right footer
\renewcommand{\headrulewidth}{0pt} % Remove header underlines
\renewcommand{\footrulewidth}{0pt} % Remove footer underlines
\setlength{\headheight}{0pt} % Customize the height of the header

\DeclareFloatingEnvironment[fileext=lod]{diagram}

\numberwithin{equation}{section} % Number equations within sections (i.e. 1.1, 1.2, 2.1, 2.2 instead of 1, 2, 3, 4)
\numberwithin{figure}{section} % Number figures within sections (i.e. 1.1, 1.2, 2.1, 2.2 instead of 1, 2, 3, 4)
\numberwithin{table}{section} % Number tables within sections (i.e. 1.1, 1.2, 2.1, 2.2 instead of 1, 2, 3, 4)

\graphicspath{{./figures/}}
%\setlength\parindent{0pt} % Removes all indentation from paragraphs - comment this line for an assignment with lots of text

\makeatletter
	\newcommand*\variableheghtrulefill[1][.4\p@]
	{%
		\leavevmode
		\leaders \hrule \@height #1\relax \hfill
		\null
	}
\makeatother

\definecolor{solBase03}{RGB}{000,043,054}
\definecolor{solBase02}{RGB}{007,054,066}
\definecolor{solBase01}{RGB}{088,110,117}
\definecolor{solBase00}{RGB}{101,123,131}
\definecolor{solBase0}{RGB}{131,148,150}
\definecolor{solBase1}{RGB}{147,161,161}
\definecolor{solBase2}{RGB}{238,232,213}
\definecolor{solBase3}{RGB}{253,246,227}
\definecolor{solYellow}{RGB}{181,137,000}
\definecolor{solOrange}{RGB}{203,075,022}
\definecolor{solRed}{RGB}{220,050,047}
\definecolor{solMagenta}{RGB}{211,054,130}
\definecolor{solViolet}{RGB}{108,113,196}
\definecolor{solBlue}{RGB}{038,139,210}
\definecolor{solCyan}{RGB}{042,161,152}
\definecolor{solGreen}{RGB}{133,153,000}

\lstdefinestyle{mystyle}{
	% To Match
	sensitive=true,	
	%
	% Add border
	frame=lines,
	%
	% Add Margin
	xleftmargin=\parindent,
	%
	% Put extra space under caption
 belowcaptionskip=1\baselineskip,
 %
 % Colors:
	backgroundcolor=\color{solBase3},
 basicstyle=\color{solBase00}\footnotesize,
 keywordstyle=\color{solCyan},
 commentstyle=\color{solBase1},
 stringstyle=\color{solBlue},
 numberstyle=\color{solViolet},
 identifierstyle=\color{solBase00},
 %
 % Formatting Options
 breakatwhitespace=false,
 breaklines=true, % break long lines
 captionpos=b,
 keepspaces=true,
 numbers=left,
 numbersep=5pt,
 showspaces=false,
 showstringspaces=false,
 showtabs=false,
 tabsize=4
}

\lstdefinestyle{smallstyle}{
	% To Match
	sensitive=true,	
	%
	% Add border
	frame=lines,
	%
	% Add Margin
	xleftmargin=\parindent,
	%
	% Put extra space under caption
 belowcaptionskip=1\baselineskip,
 %
 % Colors:
	backgroundcolor=\color{solBase3},
 basicstyle=\color{solBase00}\scriptsize,
 keywordstyle=\color{solCyan},
 commentstyle=\color{solBase1},
 stringstyle=\color{solBlue},
 numberstyle=\color{solViolet},
 identifierstyle=\color{solBase00},
 %
 % Formatting Options
 breakatwhitespace=false,
 breaklines=false, % break long lines
 captionpos=b,
 keepspaces=true,
 numbers=left,
 numbersep=5pt,
 showspaces=false,
 showstringspaces=false,
 showtabs=false,
 tabsize=2
}
 
\lstset{style=mystyle}

%----------------------------------------------------------------------------------------
%	USEFUL COMMANDS
%----------------------------------------------------------------------------------------
%	\makebox[\textwidth][c]{\includegraphics[width=.7\paperwidth]{p2-table}}

%	\newgeometry{top=.75in, bottom=.75in, left=.25in,right=.25in}
%	\newgeometry{top=.75in, bottom=.75in, left=1.25in,right=1.25in}

%	\lstinputlisting[firstline=0, language=C, style=mystyle]{CMPSC360_Homework.cpp}

%\Tree
%	[.<root> [.<left> ][.<middle> ][.<right> ]]

%----------------------------------------------------------------------------------------
%	TITLE SECTION
%----------------------------------------------------------------------------------------

\newcommand{\horrule}[1]{\rule{\linewidth}{#1}} % Create horizontal rule command with 1 argument of height
% \title{Template: Homework 1}
\title{	
\normalfont \normalsize 
%\textsc{Rutgers University, Real Analysis I} \\ [25pt] % Your university, school and/or department name(s)
\horrule{0.5pt} \\[0.4cm] % Thin top horizontal rule
\huge STAT 461: Project \\ % The assignment title
\horrule{2pt} \\[0.5cm] % Thick bottom horizontal rule
}

\author{\textbf{\underline{Name:}}Kyle Salitrik | \textit{\textbf{\underline{ID\#:}} 997543474} | \textit{\textbf{\underline{PSU ID:}} kps168}} % Your name

\date{\normalsize\today} % Today's date or a custom date

\begin{document}

\maketitle % Print the title

%----------------------------------------------------------------------------------------
%	Data Gathering
%----------------------------------------------------------------------------------------
\newgeometry{top=.75in, bottom=.75in, left=1.25in,right=1.25in}
\section*{\variableheghtrulefill[.25ex]\quad Defining Experiment \& Gathering Data \quad\variableheghtrulefill[.25ex]}
The experiment consisted of baking cookies and measuring the diameter of the finished product. The factors that were chosen were to use melted butter or creamed butter and to chill half of the dough before baking. The recipe was measured out into to separate mixing bowls to make the dough instead of mixing all of the dry ingredients and separating afterwards to ensure that the same amount of each ingredient went into the two dough mixtures. This recipe was used: 

\url{https://joyfoodsunshine.com/the-most-amazing-chocolate-chip-cookies/}

The following table outlines the factors that were considered, their levels, and how they are associated. Originally tray position was not considered as a factor, however, when putting the trays into the oven, only one tray would fit on the top rack so the difference in top and bottom racks was added after the fact.

The purpose of the experiment was to see if the melted butter, which creates a more liquidus mixture, would cause the cookies to become larger in diameter. Similarly, one would expect that the chilled dough may affect the cookie size and shape.

\begin{tabularx}{\textwidth}{| l | X | l | l |}
	\hline
	\textbf{Factor}   & \textbf{Levels}                                  & \textbf{Fixed?} & \textbf{Nested/Crossed}      \\ \hline
	Butter (b)        & Melted (M), Creamed (C)                          & Fixed           & Crossed with Dough \& Tray   \\ \hline
	Dough Temp (d)    & Refrigerated (R), \newline  NOT Refrigerated (N) & Fixed           & Crossed with Butter \& Tray  \\ \hline
	Tray Position (t) & Top (T), Bottom (B)                              & Fixed           & Crossed with Butter \& Dough \\ \hline
\end{tabularx}

Below are images of the butter, sugar, and brown sugar created by creaming the butter and sugars together vs simply mixing in melted butter.

\begin{figure}[H]
	\begin{subfigure}{.5\textwidth}
		\centering
		\includegraphics[width=.8\linewidth]{butter_creamed}
		\caption*{Creamed Butter Mixture}
	\end{subfigure}%
	\begin{subfigure}{.5\textwidth}
		\centering
		\includegraphics[width=.8\linewidth]{butter_melted}
		\caption*{Melted Butter Mixture}
	\end{subfigure}
\end{figure}

After the dough was made, each was weighed separately to ensure they were the same weight (753 grams). Afterwards, each type of dough was split in half by weight (377 grams) and placed into 2 separate bowls, covered with plastic wrap, and marked. One bowl from each type of dough was placed in the refrigerator for 2 hours.

\begin{figure}[H]
	\begin{subfigure}{.33\textwidth}
		\centering
		\includegraphics[width=.8\linewidth]{weight_full}
		\caption*{Full Weight of Dough}
	\end{subfigure}%
	\begin{subfigure}{.33\textwidth}
		\centering
		\includegraphics[width=.8\linewidth]{weight_split}
		\caption*{Weight of Dough After Separating}
	\end{subfigure}
	\begin{subfigure}{.33\textwidth}
		\centering
		\includegraphics[width=.8\linewidth]{dough_roomTemp}
		\caption*{Marked Bowls at Room Temp}
	\end{subfigure}
\end{figure}



Using the possible combinations of butter and dough temperature, R was used draw 2 samples for position assignment for each tray. The following is the output from this procedure.

\newpage
\lstinputlisting[language=, caption=Randomized Tray Locations]{listings/project-TrayLocations.txt}

The cookie dough was placed on 2 baking sheets using the above samples as a guide and marked 1-12 on each sheet. To ensure a mostly consistent cookie size, a dough scoop was used and leveled off for each cookie.

\begin{figure}[H]
	\begin{subfigure}{.5\textwidth}
		\centering
		\includegraphics[width=.8\linewidth]{cookies_tray2}
		\caption*{Tray 1}
	\end{subfigure}%
	\begin{subfigure}{.5\textwidth}
		\centering
		\includegraphics[width=.8\linewidth]{cookies_tray2}
		\caption*{Tray 2}
	\end{subfigure}
\end{figure}

The cookies were baked for 9 minutes, removed from the oven and the largest diameter was measured using a pair of calipers.

\begin{figure}[H]
	\centering
	\includegraphics[width=.8\linewidth]{cookies_finished}
	\caption*{Creamed Butter Mixture}
\end{figure}


%----------------------------------------------------------------------------------------
%	Data Plotting
%----------------------------------------------------------------------------------------
\section*{\variableheghtrulefill[.25ex]\quad Data Plots \quad\variableheghtrulefill[.25ex]}
\begin{figure}[H]
	\makebox[\textwidth][c]{\includegraphics[width=.7\paperwidth]{boxplot-individuals}}
	\caption*{Figure 1: Individual Factor Effects on Cookie Size}
\end{figure}
\begin{figure}[H]
	\makebox[\textwidth][c]{\includegraphics[width=.7\paperwidth]{boxplot-pairs}}
	\caption*{Figure 2: Pairwise Factor Effects on Cookie Size}
\end{figure}
\begin{figure}[H]
	\makebox[\textwidth][c]{\includegraphics[width=.7\paperwidth]{boxplot-full}}
	\caption*{Figure 3: Full Factor Effects on Cookie Size}
\end{figure}

%----------------------------------------------------------------------------------------
%	Models
%----------------------------------------------------------------------------------------
\section*{\variableheghtrulefill[.25ex]\quad Model \quad\variableheghtrulefill[.25ex]}
Using the full model with all interactions, we can obtain the following model:
\begin{flalign*}
	Y_{b,d,t,c} & = \alpha_{b} + \beta_{d} + \gamma_{t} + \left(\alpha\beta\right)_{b,d} + \left(\alpha\gamma\right)_{b,t} + \left(\beta\gamma\right)_{dt} + \epsilon_{b,d,t,c}& \\
	\epsilon_{b,d,t,c} & \sim N(0, \sigma^2) &
\end{flalign*}

If one were to only examine the main effects, the ME model is as follows: 
\begin{flalign*}
	Y_{b,d,t,c} & = \alpha_{b} + \beta_{d} + \gamma_{t} + \epsilon_{b,d,t,c}& \\
	\epsilon_{b,d,t,c} & \sim N(0, \sigma^2) &
\end{flalign*}

%----------------------------------------------------------------------------------------
%	Analysis
%----------------------------------------------------------------------------------------
\section*{\variableheghtrulefill[.25ex]\quad R Analysis \quad\variableheghtrulefill[.25ex]}
Looking at the ANOVA table for the full model, we can see that the only truly significant factor is the interaction between the type of butter used and the tray position. The chilled dough and butter type interaction is significant if we relax our alpha value to 0.1.
\newpage
\lstinputlisting[language=, caption=Full Model ANOVA Table]{listings/project-FullAnova.txt}

For robustness, examining the full model yields the same result that the individual factors are not enough to result in a difference in the size of the cookies.

\lstinputlisting[language=, caption=Main Effects Model ANOVA Table]{listings/project-Main.txt}

Moving to check our normality assumptions, the below Q-Q and Residual vs Fitted Plots. The residual vs fitted plot shows relatively constant variance and there is nothing significantly out of place in the Q-Q plot to indicate that our normality assumptions are violated. The tails are both a bit skewed but the majority of the data is within a reasonable variance. Performing a log and square-root transformation on the data yielded minimal change in the normality plots, so the unaltered model was used.

\begin{figure}[H]
	\makebox[\textwidth][c]{\includegraphics[width=.7\paperwidth]{normality-residuals}}
	\caption*{Figure 4: Residual vs Fitted Plot for Full Model}
\end{figure}
\begin{figure}[H]
	\makebox[\textwidth][c]{\includegraphics[width=.7\paperwidth]{normality-qq}}
	\caption*{Figure 5: Q-Q Plot for Full Model}
\end{figure}

Finally, because the only significant terms are interaction terms, an interaction plot was created for the two interaction pairs. These plots are below:

\begin{figure}[H]
	\makebox[\textwidth][c]{\includegraphics[width=.7\paperwidth]{interaction-tray-butter}}
	\caption*{Figure 6: (Tray Position:Butter Type) Interaction Plot}
\end{figure}
\begin{figure}[H]
	\makebox[\textwidth][c]{\includegraphics[width=.7\paperwidth]{interaction-chilled-butter}}
	\caption*{Figure 7: (Dough Temperature:Butter Type) Interaction Plot}
\end{figure}

%----------------------------------------------------------------------------------------
%	Conclusions
%----------------------------------------------------------------------------------------
\section*{\variableheghtrulefill[.25ex]\quad Conclusions \quad\variableheghtrulefill[.25ex]}
Based on the R analysis, we can see that no individual factor significantly influenced the size of the cookies, however the interaction between type of butter used and position of the tray on the top or bottom rack did significantly impact the cookies. For the melted butter, putting the cookies on the top rack allowed them to spread wider than the creamed butter cookies while the opposite was true for placing the tray on the bottom rack. The dough temperature had a similar pair of interactions between the two types of cookies.

One reason that using the bottom rack, in general, gave smaller cookies is that the bases of the cookies cooked a lot faster which caused them to grow taller (or stay taller) as the base solidified faster. As for improvements to the experiment, knowing that both trays would not fit on the top rack ahead of time and randomly assigning that using R would have been a better idea than just picking one to move. Another improvement would be to measure the cookies by weight instead of using a dough scoop. A final change would be to measure out the individual cookies first and then simply transfer them to the trays after chilling instead of having to form each cookie as it would allow less time for the dough to reach room temperature before putting them into the oven.



\newpage
\section*{Data Appendix}
\begin{table}[H]
	\centering
	\begin{tabular}{|c|c|c|c|}
		\hline
		\textbf{Size (mm)} & \textbf{Butter} & \textbf{Dough Temperature} & \textbf{Tray Position} \\ \hline
		85.61              & C               & N                          & B                      \\ \hline
		87.56              & C               & N                          & B                      \\ \hline
		83.03              & M               & R                          & B                      \\ \hline
		87.78              & C               & N                          & B                      \\ \hline
		81.78              & M               & R                          & B                      \\ \hline
		83.70              & C               & R                          & B                      \\ \hline
		84.32              & C               & R                          & B                      \\ \hline
		87.83              & M               & R                          & B                      \\ \hline
		82.28              & M               & N                          & B                      \\ \hline
		81.06              & M               & N                          & B                      \\ \hline
		79.60              & C               & R                          & B                      \\ \hline
		84.38              & M               & N                          & B                      \\ \hline
		91.32              & C               & N                          & T                      \\ \hline
		77.38              & C               & N                          & T                      \\ \hline
		73.49              & C               & R                          & T                      \\ \hline
		78.23              & C               & R                          & T                      \\ \hline
		86.69              & M               & R                          & T                      \\ \hline
		87.93              & M               & N                          & T                      \\ \hline
		86.72              & C               & N                          & T                      \\ \hline
		87.45              & M               & N                          & T                      \\ \hline
		91.57              & M               & R                          & T                      \\ \hline
		92.12              & M               & R                          & T                      \\ \hline
		92.17              & M               & N                          & T                      \\ \hline
		88.42              & C               & R                          & T                      \\ \hline
	\end{tabular}
	\caption*{Table of Collected Data}
\end{table}

\newpage
\section*{Code Appendix}
\lstinputlisting[language=R]{project.R}

\end{document}