\documentclass[paper=a4, fontsize=11pt]{scrartcl} % A4 paper and 11pt font size

%----------------------------------------------------------------------------------------
%	PACKAGES
%----------------------------------------------------------------------------------------
\usepackage[T1]{fontenc} % Use 8-bit encoding that has 256 glyphs
\usepackage{fourier} % Use the Adobe Utopia font for the document - comment this line to return to the LaTeX default
\usepackage[english]{babel} % English language/hyphenation
\usepackage{amsmath,amsfonts,amsthm} % Math packages
\usepackage{sectsty} % Allows customizing section commands
\usepackage{fancyhdr} % Custom headers and footers
\usepackage{xcolor} % Allows 0-255 RGB values
\usepackage{tabularx, outlines, framed, varwidth, enumitem, graphicx, listings, color, qtree, float, subcaption, newfloat}
\usepackage[left=0.5in, right=0.5in, top=3in, bottom=.25in]{geometry}
\geometry{}

%----------------------------------------------------------------------------------------
%	SET CUSTOMIZATIONS AND FUNCTIONS
%----------------------------------------------------------------------------------------
\sectionfont{\centering \normalfont\scshape} % Make all sections centered, the default font and small caps
\pagestyle{fancyplain} % Makes all pages in the document conform to the custom headers and footers
\fancyhead{} % No page header - if you want one, create it in the same way as the footers below
\fancyfoot[L]{} % Empty left footer
\fancyfoot[C]{} % Empty center footer
\fancyfoot[R]{\thepage} % Page numbering for right footer
\renewcommand{\headrulewidth}{0pt} % Remove header underlines
\renewcommand{\footrulewidth}{0pt} % Remove footer underlines
\setlength{\headheight}{0pt} % Customize the height of the header

\DeclareFloatingEnvironment[fileext=lod]{diagram}

\numberwithin{equation}{section} % Number equations within sections (i.e. 1.1, 1.2, 2.1, 2.2 instead of 1, 2, 3, 4)
\numberwithin{figure}{section} % Number figures within sections (i.e. 1.1, 1.2, 2.1, 2.2 instead of 1, 2, 3, 4)
\numberwithin{table}{section} % Number tables within sections (i.e. 1.1, 1.2, 2.1, 2.2 instead of 1, 2, 3, 4)

\graphicspath{{./figures/}}
%\setlength\parindent{0pt} % Removes all indentation from paragraphs - comment this line for an assignment with lots of text

\makeatletter
	\newcommand*\variableheghtrulefill[1][.4\p@]
	{%
		\leavevmode
		\leaders \hrule \@height #1\relax \hfill
		\null
	}
\makeatother

\definecolor{solBase03}{RGB}{000,043,054}
\definecolor{solBase02}{RGB}{007,054,066}
\definecolor{solBase01}{RGB}{088,110,117}
\definecolor{solBase00}{RGB}{101,123,131}
\definecolor{solBase0}{RGB}{131,148,150}
\definecolor{solBase1}{RGB}{147,161,161}
\definecolor{solBase2}{RGB}{238,232,213}
\definecolor{solBase3}{RGB}{253,246,227}
\definecolor{solYellow}{RGB}{181,137,000}
\definecolor{solOrange}{RGB}{203,075,022}
\definecolor{solRed}{RGB}{220,050,047}
\definecolor{solMagenta}{RGB}{211,054,130}
\definecolor{solViolet}{RGB}{108,113,196}
\definecolor{solBlue}{RGB}{038,139,210}
\definecolor{solCyan}{RGB}{042,161,152}
\definecolor{solGreen}{RGB}{133,153,000}

\lstdefinestyle{mystyle}{
	% To Match
	sensitive=true,	
	%
	% Add border
	frame=lines,
	%
	% Add Margin
	xleftmargin=\parindent,
	%
	% Put extra space under caption
    belowcaptionskip=1\baselineskip,
    %
    % Colors:
	backgroundcolor=\color{solBase3},
    basicstyle=\color{solBase00}\footnotesize,
    keywordstyle=\color{solCyan},
    commentstyle=\color{solBase1},
    stringstyle=\color{solBlue},
    numberstyle=\color{solViolet},
    identifierstyle=\color{solBase00},
    %
    % Formatting Options
    breakatwhitespace=false,
    breaklines=true, % break long lines
    captionpos=b,
    keepspaces=true,
    numbers=left,
    numbersep=5pt,
    showspaces=false,
    showstringspaces=false,
    showtabs=false,
    tabsize=4
}

\lstdefinestyle{smallstyle}{
	% To Match
	sensitive=true,	
	%
	% Add border
	frame=lines,
	%
	% Add Margin
	xleftmargin=\parindent,
	%
	% Put extra space under caption
    belowcaptionskip=1\baselineskip,
    %
    % Colors:
	backgroundcolor=\color{solBase3},
    basicstyle=\color{solBase00}\scriptsize,
    keywordstyle=\color{solCyan},
    commentstyle=\color{solBase1},
    stringstyle=\color{solBlue},
    numberstyle=\color{solViolet},
    identifierstyle=\color{solBase00},
    %
    % Formatting Options
    breakatwhitespace=false,
    breaklines=false, % break long lines
    captionpos=b,
    keepspaces=true,
    numbers=left,
    numbersep=5pt,
    showspaces=false,
    showstringspaces=false,
    showtabs=false,
    tabsize=2
}
 
\lstset{style=mystyle}

%----------------------------------------------------------------------------------------
%	USEFUL COMMANDS
%----------------------------------------------------------------------------------------
%	\makebox[\textwidth][c]{\includegraphics[width=.9\paperwidth]{p2-table}}

%	\newgeometry{top=.75in, bottom=.75in, left=.25in,right=.25in}
%	\newgeometry{top=.75in, bottom=.75in, left=1.25in,right=1.25in}

%	\lstinputlisting[firstline=0, language=C, style=mystyle]{CMPSC360_Homework.cpp}

%\Tree
%	[.<root> [.<left> ][.<middle> ][.<right> ]]

%----------------------------------------------------------------------------------------
%	TITLE SECTION
%----------------------------------------------------------------------------------------

\newcommand{\horrule}[1]{\rule{\linewidth}{#1}} % Create horizontal rule command with 1 argument of height
% \title{Template: Homework 1}
\title{	
\normalfont \normalsize 
%\textsc{Rutgers University, Real Analysis I} \\ [25pt] % Your university, school and/or department name(s)
\horrule{0.5pt} \\[0.4cm] % Thin top horizontal rule
\huge STAT 480: Homework 7 \\ % The assignment title
\horrule{2pt} \\[0.5cm] % Thick bottom horizontal rule
}

\author{\textbf{\underline{Name:}}Kyle Salitrik | \textit{\textbf{\underline{ID\#:}} 997543474} | \textit{\textbf{\underline{PSU ID:}} kps168}} % Your name

\date{\normalsize\today} % Today's date or a custom date

\begin{document}

\maketitle % Print the title

%----------------------------------------------------------------------------------------
%	PROBLEM 1
%----------------------------------------------------------------------------------------
\newgeometry{top=.75in, bottom=.75in, left=1.25in,right=1.25in}
\section*{\variableheghtrulefill[.25ex]\quad Problem 1 \quad\variableheghtrulefill[.25ex]}
\begin{table}[H]
	\centering
	\caption{Input Buffer During Iteration 1}
	\begin{tabular}{|c|c|c|c|c|c|c|c|c|c|c|c|c|}
		\hline
		1 & 2 & 3 & 4 & 5 & 6 & 7 & 8 & 9 & 10 & 11 & 12 & 13 \dots \\ \hline
		S & m & i & t & h & ~ & 1 & 0 & 0 & ~ & ~ & 9 & O\\ \hline
	\end{tabular}
\end{table}

%----------------------------------------------------------------------------------------
%	PROBLEM 2
%----------------------------------------------------------------------------------------
\section*{\variableheghtrulefill[.25ex]\quad Problem 2 \quad\variableheghtrulefill[.25ex]}
\begin{table}[H]
	\centering
	\caption{Program Data Vector Before Iteration 1}
	\begin{tabular}{|c|c|c|c|c|c|}
		\hline
		\_N\_ & \_ERROR\_ & student & g1 & g2 & avg \\ \hline
		1 & 0 & & . & . & . \\ \hline
	\end{tabular}
\end{table}

%----------------------------------------------------------------------------------------
%	PROBLEM 3
%----------------------------------------------------------------------------------------
\section*{\variableheghtrulefill[.25ex]\quad Problem 3 \quad\variableheghtrulefill[.25ex]}
\begin{table}[H]
	\centering
	\caption{Program Data Vector After Iteration 1}
	\begin{tabular}{|c|c|c|c|c|c|}
		\hline
		\_N\_ & \_ERROR\_ & student & g1 & g2 & avg \\ \hline
		1 & 1 & Smith & 100 & . & . \\ \hline
	\end{tabular}
\end{table}


%----------------------------------------------------------------------------------------
%	PROBLEM 4
%----------------------------------------------------------------------------------------
\section*{\variableheghtrulefill[.25ex]\quad Problem 4 \quad\variableheghtrulefill[.25ex]}
\begin{table}[H]
	\centering
	\caption{Grades Dataset}
	\begin{tabular}{|c|c|c|c|}
		\hline
		student & g1 & g2 & avg \\ \hline
		Smith & 100 & . & . \\ \hline
	\end{tabular}
\end{table}


%----------------------------------------------------------------------------------------
%	PROBLEM 5
%----------------------------------------------------------------------------------------
\newgeometry{top=.75in, bottom=.75in, left=1.25in,right=1.25in}
\section*{\variableheghtrulefill[.25ex]\quad Problem 5 \quad\variableheghtrulefill[.25ex]}
\begin{table}[H]
	\centering
	\caption{Input Buffer During Iteration 2}
	\begin{tabular}{|c|c|c|c|c|c|c|c|c|c|c|c|c|}
		\hline
		1 & 2 & 3 & 4 & 5 & 6 & 7 & 8 & 9 & 10 & 11 & 12 & 13 \dots \\ \hline
		J & o & n & e & s & ~ & ~ & 8 & 4 & ~ & ~ & 8 & 6\\ \hline
	\end{tabular}
\end{table}

%----------------------------------------------------------------------------------------
%	PROBLEM 6
%----------------------------------------------------------------------------------------
\section*{\variableheghtrulefill[.25ex]\quad Problem 6 \quad\variableheghtrulefill[.25ex]}
\begin{table}[H]
	\centering
	\caption{Program Data Vector Before Iteration 2}
	\begin{tabular}{|c|c|c|c|c|c|}
		\hline
		\_N\_ & \_ERROR\_ & student & g1 & g2 & avg \\ \hline
		2 & 0 & & . & . & . \\ \hline
	\end{tabular}
\end{table}

%----------------------------------------------------------------------------------------
%	PROBLEM 7
%----------------------------------------------------------------------------------------
\section*{\variableheghtrulefill[.25ex]\quad Problem 7 \quad\variableheghtrulefill[.25ex]}
\begin{table}[H]
	\centering
	\caption{Program Data Vector After Iteration 2}
	\begin{tabular}{|c|c|c|c|c|c|}
		\hline
		\_N\_ & \_ERROR\_ & student & g1 & g2 & avg \\ \hline
		2 & 0 & Jones & 84 & 86 & 85 \\ \hline
	\end{tabular}
\end{table}

%----------------------------------------------------------------------------------------
%	PROBLEM 8
%----------------------------------------------------------------------------------------
\section*{\variableheghtrulefill[.25ex]\quad Problem 8 \quad\variableheghtrulefill[.25ex]}
The largest value N will take on is 5.

%----------------------------------------------------------------------------------------
%	PROBLEM 9
%----------------------------------------------------------------------------------------
\section*{\variableheghtrulefill[.25ex]\quad Problem 9 \quad\variableheghtrulefill[.25ex]}
\begin{table}[H]
	\centering
	\caption{Grades Dataset}
	\begin{tabular}{|c|c|c|c|}
		\hline
		student & g1 & g2 & avg \\ \hline
		Smith & 100 & . & . \\ \hline
		Jones & 84 & 86 & 85 \\ \hline
		Black & 95 & 75 & 85 \\ \hline
		White & 90 & . & . \\ \hline
	\end{tabular}
\end{table}

%----------------------------------------------------------------------------------------
%	PROBLEM 10
%----------------------------------------------------------------------------------------
\section*{\variableheghtrulefill[.25ex]\quad Problem 10 \quad\variableheghtrulefill[.25ex]}
No. There should be no warning or error messages. However, you should get notes for the invalid value for grade 2 in the first record because 9O is not a valid number. You should also get a warning about missing values for the calculations that have missing values (records 1 and 4).
\end{document}