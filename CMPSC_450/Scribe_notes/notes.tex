\documentclass[a4paper, 11pt]{article}
\usepackage{comment} % enables the use of multi-line comments (\ifx \fi) 
\usepackage{fullpage} % changes the margin
\usepackage{color, listings, graphicx,float, booktabs, tabularx, multirow, outlines}
\usepackage[colorlinks=true, urlcolor=blue]{hyperref}

\definecolor{codegreen}{rgb}{0,0.6,0}
\definecolor{codegray}{rgb}{0.5,0.5,0.5}
\definecolor{codepurple}{rgb}{0.58,0,0.82}
\definecolor{backcolour}{rgb}{0.95,0.95,0.92}
 
\lstdefinestyle{mystyle}{
    backgroundcolor=\color{backcolour},   
    commentstyle=\color{codegreen},
    keywordstyle=\color{magenta},
    numberstyle=\tiny\color{codegray},
    stringstyle=\color{codepurple},
    basicstyle=\footnotesize,
    breakatwhitespace=false,         
    breaklines=true,                 
    captionpos=b,                    
    keepspaces=true,                 
    numbers=left,                    
    numbersep=5pt,                  
    showspaces=false,                
    showstringspaces=false,
    showtabs=false,                  
    tabsize=2
}
 
\lstset{style=mystyle}

\begin{document}
\graphicspath{{./figures/}}
\noindent
\large\textbf{Scribe Notes 9/21/2017} \\
\normalsize CMPSC 450\\
\large{Kyle Salitrik} \hfill 

%%%%%%%%%%%%%%%%%%%%%%%%%%%%%%%%%%%%%%%%%%%%%%%%%%%%%%%%%%%%%%%%%%%%
%%% SECTION: OpenMP
%%%%%%%%%%%%%%%%%%%%%%%%%%%%%%%%%%%%%%%%%%%%%%%%%%%%%%%%%%%%%%%%%%%%
\section*{OpenMP}

\begin{outline}
	\1 Private Variables
		\2 Firstprivate - initializes variables from serial part of the code
		\2 Lastprivate - thread that executes the final loop copies it's value to the master thread
	\1 Barriers 
		\2 Using \textit{\#pragma omp critical} causes each thread to perform computation sequentially instead of in parallel for structured block
		\2 Each structured block has implicit barriers at the end of the block where finished threads will hang until all threads reach that point
		\2 The \textit{nowait} flag prevents the threads from waiting at the barrier. Caution should be used if the output from the nowait section is needed further in the code.
	\1 Reduction
		\2 Reduction actions (+:sum, subtract, etc.) can be used in order to accumulate values from threads.
	\1 For Loops
		\2 \textit{\#pragma omp for}$\rightarrow$ loop will be executed concurrently by P threads. Removes need to skip by number of threads assigned to a parallel block.
		\2 \textit{\#pragma omp parallel}for $\rightarrow$ loop will execute entirely in all P threads (causes P times the amount of work).
		\2 Loop variables must be an integer and control parameters are the same for all threads.
		\2 Branching out of loops is prohibited (this would cause more than one control parameter).
	
	\1 Scheduling
		\2 Static scheduling is default and breaks up a loop in a logical fashion, such as n/p consecutive values per thread.
		\2 \textit{schedule(dynamic)} 
		
\end{outline}

%%%%%%%%%%%%%%%%%%%%%%%%%%%%%%%%%%%%%%%%%%%%%%%%%%%%%%%%%%%%%%%%%%%%
%%% SECTION: ICS-ACI
%%%%%%%%%%%%%%%%%%%%%%%%%%%%%%%%%%%%%%%%%%%%%%%%%%%%%%%%%%%%%%%%%%%%
\section*{ICS-ACI}

\begin{outline}
	\1 Login via SSH or EoD (Exceed on Demand, ACI-I only) with 2-factor authentication
		\2 SSH to ACI-B: ssh user@aci-b.aci.ics.psu.edu
		\2 SSH to ACI-I: ssh user@aci-i.aci.ics.psu.edu
	\1 Help email: iask@ics.psu.edu
	\1 ACI-I vs ACI-B
		\2 ACI-I (Interactive) nodes are shared with other people
			\3 Limited to 4 processors at 12 total user hours with 48GB of memory.
		\2 ACI-B (Batch) nodes are allocated to individuals based on need and open nodes.
			\3 Limited to 20 processors
			\3 Shell scripts are used to perform batch programs via submission
			\3 Modules are called in the script to load software to the stack
	\1 Environments
		\2 Home: 10GB backed up
		\2 Work: 128GB backed up (faster than home)
		\2 Scratch: 1 million file limit (no size limit)
	\1 Filezilla/WinSCP can be used for file transfer to the ACI servers
		\2 url: datamgr.aci.ics.psu.edu
\end{outline}

%%%%%%%%%%%%%%%%%%%%%%%%%%%%%%%%%%%%%%%%%%%%%%%%%%%%%%%%%%%%%%%%%%%%
%%% SECTION: HW Assign
%%%%%%%%%%%%%%%%%%%%%%%%%%%%%%%%%%%%%%%%%%%%%%%%%%%%%%%%%%%%%%%%%%%%
\section*{Homework}
\begin{outline}
	\1 Due Oct 3.
	\1 Write two implementations of the PRAM summation algorithm using OpenMP to sum N elements.
		\2 The first version should not use the reduction clauses from OpenMP, the second should use the clauses.
\end{outline}


\end{document}